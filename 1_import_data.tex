% Options for packages loaded elsewhere
\PassOptionsToPackage{unicode}{hyperref}
\PassOptionsToPackage{hyphens}{url}
%
\documentclass[
]{article}
\usepackage{amsmath,amssymb}
\usepackage{iftex}
\ifPDFTeX
  \usepackage[T1]{fontenc}
  \usepackage[utf8]{inputenc}
  \usepackage{textcomp} % provide euro and other symbols
\else % if luatex or xetex
  \usepackage{unicode-math} % this also loads fontspec
  \defaultfontfeatures{Scale=MatchLowercase}
  \defaultfontfeatures[\rmfamily]{Ligatures=TeX,Scale=1}
\fi
\usepackage{lmodern}
\ifPDFTeX\else
  % xetex/luatex font selection
\fi
% Use upquote if available, for straight quotes in verbatim environments
\IfFileExists{upquote.sty}{\usepackage{upquote}}{}
\IfFileExists{microtype.sty}{% use microtype if available
  \usepackage[]{microtype}
  \UseMicrotypeSet[protrusion]{basicmath} % disable protrusion for tt fonts
}{}
\makeatletter
\@ifundefined{KOMAClassName}{% if non-KOMA class
  \IfFileExists{parskip.sty}{%
    \usepackage{parskip}
  }{% else
    \setlength{\parindent}{0pt}
    \setlength{\parskip}{6pt plus 2pt minus 1pt}}
}{% if KOMA class
  \KOMAoptions{parskip=half}}
\makeatother
\usepackage{xcolor}
\usepackage[margin=1in]{geometry}
\usepackage{color}
\usepackage{fancyvrb}
\newcommand{\VerbBar}{|}
\newcommand{\VERB}{\Verb[commandchars=\\\{\}]}
\DefineVerbatimEnvironment{Highlighting}{Verbatim}{commandchars=\\\{\}}
% Add ',fontsize=\small' for more characters per line
\usepackage{framed}
\definecolor{shadecolor}{RGB}{248,248,248}
\newenvironment{Shaded}{\begin{snugshade}}{\end{snugshade}}
\newcommand{\AlertTok}[1]{\textcolor[rgb]{0.94,0.16,0.16}{#1}}
\newcommand{\AnnotationTok}[1]{\textcolor[rgb]{0.56,0.35,0.01}{\textbf{\textit{#1}}}}
\newcommand{\AttributeTok}[1]{\textcolor[rgb]{0.13,0.29,0.53}{#1}}
\newcommand{\BaseNTok}[1]{\textcolor[rgb]{0.00,0.00,0.81}{#1}}
\newcommand{\BuiltInTok}[1]{#1}
\newcommand{\CharTok}[1]{\textcolor[rgb]{0.31,0.60,0.02}{#1}}
\newcommand{\CommentTok}[1]{\textcolor[rgb]{0.56,0.35,0.01}{\textit{#1}}}
\newcommand{\CommentVarTok}[1]{\textcolor[rgb]{0.56,0.35,0.01}{\textbf{\textit{#1}}}}
\newcommand{\ConstantTok}[1]{\textcolor[rgb]{0.56,0.35,0.01}{#1}}
\newcommand{\ControlFlowTok}[1]{\textcolor[rgb]{0.13,0.29,0.53}{\textbf{#1}}}
\newcommand{\DataTypeTok}[1]{\textcolor[rgb]{0.13,0.29,0.53}{#1}}
\newcommand{\DecValTok}[1]{\textcolor[rgb]{0.00,0.00,0.81}{#1}}
\newcommand{\DocumentationTok}[1]{\textcolor[rgb]{0.56,0.35,0.01}{\textbf{\textit{#1}}}}
\newcommand{\ErrorTok}[1]{\textcolor[rgb]{0.64,0.00,0.00}{\textbf{#1}}}
\newcommand{\ExtensionTok}[1]{#1}
\newcommand{\FloatTok}[1]{\textcolor[rgb]{0.00,0.00,0.81}{#1}}
\newcommand{\FunctionTok}[1]{\textcolor[rgb]{0.13,0.29,0.53}{\textbf{#1}}}
\newcommand{\ImportTok}[1]{#1}
\newcommand{\InformationTok}[1]{\textcolor[rgb]{0.56,0.35,0.01}{\textbf{\textit{#1}}}}
\newcommand{\KeywordTok}[1]{\textcolor[rgb]{0.13,0.29,0.53}{\textbf{#1}}}
\newcommand{\NormalTok}[1]{#1}
\newcommand{\OperatorTok}[1]{\textcolor[rgb]{0.81,0.36,0.00}{\textbf{#1}}}
\newcommand{\OtherTok}[1]{\textcolor[rgb]{0.56,0.35,0.01}{#1}}
\newcommand{\PreprocessorTok}[1]{\textcolor[rgb]{0.56,0.35,0.01}{\textit{#1}}}
\newcommand{\RegionMarkerTok}[1]{#1}
\newcommand{\SpecialCharTok}[1]{\textcolor[rgb]{0.81,0.36,0.00}{\textbf{#1}}}
\newcommand{\SpecialStringTok}[1]{\textcolor[rgb]{0.31,0.60,0.02}{#1}}
\newcommand{\StringTok}[1]{\textcolor[rgb]{0.31,0.60,0.02}{#1}}
\newcommand{\VariableTok}[1]{\textcolor[rgb]{0.00,0.00,0.00}{#1}}
\newcommand{\VerbatimStringTok}[1]{\textcolor[rgb]{0.31,0.60,0.02}{#1}}
\newcommand{\WarningTok}[1]{\textcolor[rgb]{0.56,0.35,0.01}{\textbf{\textit{#1}}}}
\usepackage{graphicx}
\makeatletter
\def\maxwidth{\ifdim\Gin@nat@width>\linewidth\linewidth\else\Gin@nat@width\fi}
\def\maxheight{\ifdim\Gin@nat@height>\textheight\textheight\else\Gin@nat@height\fi}
\makeatother
% Scale images if necessary, so that they will not overflow the page
% margins by default, and it is still possible to overwrite the defaults
% using explicit options in \includegraphics[width, height, ...]{}
\setkeys{Gin}{width=\maxwidth,height=\maxheight,keepaspectratio}
% Set default figure placement to htbp
\makeatletter
\def\fps@figure{htbp}
\makeatother
\setlength{\emergencystretch}{3em} % prevent overfull lines
\providecommand{\tightlist}{%
  \setlength{\itemsep}{0pt}\setlength{\parskip}{0pt}}
\setcounter{secnumdepth}{-\maxdimen} % remove section numbering
\ifLuaTeX
  \usepackage{selnolig}  % disable illegal ligatures
\fi
\IfFileExists{bookmark.sty}{\usepackage{bookmark}}{\usepackage{hyperref}}
\IfFileExists{xurl.sty}{\usepackage{xurl}}{} % add URL line breaks if available
\urlstyle{same}
\hypersetup{
  pdftitle={Import sequencing data into R},
  hidelinks,
  pdfcreator={LaTeX via pandoc}}

\title{Import sequencing data into R}
\author{}
\date{\vspace{-2.5em}}

\begin{document}
\maketitle

The First step . . .bla bla bla . . .

Let's start with the Illumina dataset

For these steps wee will need the following libraries

\begin{Shaded}
\begin{Highlighting}[]
\FunctionTok{library}\NormalTok{(}\StringTok{"tidyverse"}\NormalTok{)}
\FunctionTok{library}\NormalTok{(}\StringTok{"phyloseq"}\NormalTok{)}
\end{Highlighting}
\end{Shaded}

First import the ASV table

\begin{Shaded}
\begin{Highlighting}[]
\NormalTok{biom\_illumina}\OtherTok{\textless{}{-}} \FunctionTok{read.csv}\NormalTok{(}\StringTok{"Raw\_data/Illumina/illumina\_ASV\_table.tsv"}\NormalTok{, }
                        \AttributeTok{header=}\ConstantTok{TRUE}\NormalTok{, }\AttributeTok{sep=}\StringTok{"}\SpecialCharTok{\textbackslash{}t}\StringTok{"}\NormalTok{)}
\FunctionTok{head}\NormalTok{(biom\_illumina[,}\DecValTok{1}\SpecialCharTok{:}\DecValTok{5}\NormalTok{],}\DecValTok{10}\NormalTok{)}
\end{Highlighting}
\end{Shaded}

\begin{verbatim}
##                                 otu QC0221WK20 QC0221WK21 QC0221WK22 QC0221WK23
## 1  f556cceb09c61e1530d00b7d1aed8234          0          0          0          0
## 2  353c06adc7d2edfcfaebf59bbe3ff76e        400          0       2617          0
## 3  9b00bc511c2b9d7b833604813b0b05f2        510       1678       1870       6213
## 4  9e1dc7d16af91d5c34643b33f8f13694          0          0          0          0
## 5  d87923a926982434c811a4fc33a7942f        435       3938        574          0
## 6  9ed6ed511e00f993f42bc338c94d55eb       7513       5636       1616       2387
## 7  d11f971c2573dded4a7b813736d54cc3          0          0        930          0
## 8  a62e14ddf957e10799c77e23e1df70e9          0          0       1098       2531
## 9  1d05a08941ce5f54c3ecac0a1d850780        951       4696        932        865
## 10 da68b84ae7440726f0fa022c707fe95b          0          0          0          0
\end{verbatim}

Then import the taxonomy table

\begin{Shaded}
\begin{Highlighting}[]
\NormalTok{taxo\_illumina}\OtherTok{\textless{}{-}} \FunctionTok{read.csv}\NormalTok{(}\StringTok{"Raw\_data/Illumina/illumina\_taxonomy.tsv"}\NormalTok{, }
                        \AttributeTok{header=}\ConstantTok{TRUE}\NormalTok{, }\AttributeTok{sep=}\StringTok{"}\SpecialCharTok{\textbackslash{}t}\StringTok{"}\NormalTok{)}
\FunctionTok{head}\NormalTok{(taxo\_illumina[,}\DecValTok{1}\SpecialCharTok{:}\DecValTok{8}\NormalTok{],}\DecValTok{10}\NormalTok{)}
\end{Highlighting}
\end{Shaded}

\begin{verbatim}
##                                 otu      Kingdom      Phylum            Class
## 1  f556cceb09c61e1530d00b7d1aed8234 k__Eukaryota p__Oomycota c__Stramenopiles
## 2  353c06adc7d2edfcfaebf59bbe3ff76e k__Eukaryota p__Oomycota c__Stramenopiles
## 3  9b00bc511c2b9d7b833604813b0b05f2 k__Eukaryota p__Oomycota c__Stramenopiles
## 4  9e1dc7d16af91d5c34643b33f8f13694 k__Eukaryota p__Oomycota c__Stramenopiles
## 5  d87923a926982434c811a4fc33a7942f k__Eukaryota p__Oomycota c__Stramenopiles
## 6  9ed6ed511e00f993f42bc338c94d55eb k__Eukaryota p__Oomycota c__Stramenopiles
## 7  d11f971c2573dded4a7b813736d54cc3 k__Eukaryota p__Oomycota c__Stramenopiles
## 8  a62e14ddf957e10799c77e23e1df70e9 k__Eukaryota p__Oomycota c__Stramenopiles
## 9  1d05a08941ce5f54c3ecac0a1d850780 k__Eukaryota p__Oomycota c__Stramenopiles
## 10 da68b84ae7440726f0fa022c707fe95b k__Eukaryota p__Oomycota c__Stramenopiles
##                Order             Family               Genus
## 1  o__Peronosporales f__Peronosporaceae      g__Peronospora
## 2  o__Peronosporales f__Peronosporaceae      g__Peronospora
## 3  o__Peronosporales f__Peronosporaceae      g__Peronospora
## 4  o__Peronosporales f__Peronosporaceae           g__Bremia
## 5     o__Albuginales    f__Albuginaceae           g__Albugo
## 6  o__Peronosporales f__Peronosporaceae      g__Peronospora
## 7  o__Peronosporales f__Peronosporaceae      g__Peronospora
## 8  o__Peronosporales f__Peronosporaceae      g__Peronospora
## 9  o__Peronosporales f__Peronosporaceae g__Hyaloperonospora
## 10 o__Peronosporales f__Peronosporaceae g__Hyaloperonospora
##                          Species
## 1      s__Peronospora_manshurica
## 2      s__Peronospora_destructor
## 3      s__Peronospora_variabilis
## 4            s__Bremia_elliptica
## 5              s__Albugo_candida
## 6        s__Peronospora_aparines
## 7                               
## 8      s__Peronospora_variabilis
## 9    s__Hyaloperonospora_nesliae
## 10 s__Hyaloperonospora_brassicae
\end{verbatim}

Then define the row names from the otu column

\begin{Shaded}
\begin{Highlighting}[]
\NormalTok{biom\_illumina }\OtherTok{\textless{}{-}}\NormalTok{ biom\_illumina }\SpecialCharTok{\%\textgreater{}\%}
\NormalTok{  tibble}\SpecialCharTok{::}\FunctionTok{column\_to\_rownames}\NormalTok{(}\StringTok{"otu"}\NormalTok{) }

\NormalTok{taxo\_illumina }\OtherTok{\textless{}{-}}\NormalTok{ taxo\_illumina }\SpecialCharTok{\%\textgreater{}\%}
\NormalTok{  tibble}\SpecialCharTok{::}\FunctionTok{column\_to\_rownames}\NormalTok{(}\StringTok{"otu"}\NormalTok{) }
\end{Highlighting}
\end{Shaded}

Transform otu and taxonomy tables into matrixes

\begin{Shaded}
\begin{Highlighting}[]
\NormalTok{biom\_illumina }\OtherTok{\textless{}{-}} \FunctionTok{as.matrix}\NormalTok{(biom\_illumina)}
\NormalTok{taxo\_illumina }\OtherTok{\textless{}{-}} \FunctionTok{as.matrix}\NormalTok{(taxo\_illumina)}

\FunctionTok{class}\NormalTok{(biom\_illumina)}
\end{Highlighting}
\end{Shaded}

\begin{verbatim}
## [1] "matrix" "array"
\end{verbatim}

\begin{Shaded}
\begin{Highlighting}[]
\FunctionTok{class}\NormalTok{(taxo\_illumina) }
\end{Highlighting}
\end{Shaded}

\begin{verbatim}
## [1] "matrix" "array"
\end{verbatim}

convert to phyloseq objects

\begin{Shaded}
\begin{Highlighting}[]
\NormalTok{OTU\_illumina }\OtherTok{=} \FunctionTok{otu\_table}\NormalTok{(biom\_illumina, }\AttributeTok{taxa\_are\_rows =} \ConstantTok{TRUE}\NormalTok{)}
\NormalTok{TAX\_illumina }\OtherTok{=}\NormalTok{ phyloseq}\SpecialCharTok{::}\FunctionTok{tax\_table}\NormalTok{(taxo\_illumina)}

\NormalTok{illuminaPS }\OtherTok{\textless{}{-}} \FunctionTok{phyloseq}\NormalTok{(OTU\_illumina, TAX\_illumina)}
\NormalTok{illuminaPS}
\end{Highlighting}
\end{Shaded}

\begin{verbatim}
## phyloseq-class experiment-level object
## otu_table()   OTU Table:         [ 324 taxa and 112 samples ]
## tax_table()   Taxonomy Table:    [ 324 taxa by 7 taxonomic ranks ]
\end{verbatim}

Let's repeat for the IonTorrent dataset

\begin{Shaded}
\begin{Highlighting}[]
\NormalTok{biom\_IonTorrent}\OtherTok{\textless{}{-}} \FunctionTok{read.csv}\NormalTok{(}\StringTok{"Raw\_data/IonTorrent/iontorrent\_ASV\_table.tsv"}\NormalTok{, }
                        \AttributeTok{header=}\ConstantTok{TRUE}\NormalTok{, }\AttributeTok{sep=}\StringTok{"}\SpecialCharTok{\textbackslash{}t}\StringTok{"}\NormalTok{)}
\FunctionTok{head}\NormalTok{(biom\_IonTorrent[,}\DecValTok{1}\SpecialCharTok{:}\DecValTok{5}\NormalTok{],}\DecValTok{10}\NormalTok{)}
\end{Highlighting}
\end{Shaded}

\begin{verbatim}
##                                 otu IonT_QC0221WK20 IonT_QC0221WK23
## 1  2949b355cd00039476d9a808f2e76b8e               0               0
## 2  9d34744d5f40b8a99cc028bb4908e00b              16              27
## 3  61c226ba90e400b577fe1d486db3256d               7               0
## 4  af77980eb4e8e0abee3729db262bcdae               3               0
## 5  5f838c2663e51903bc2edc67cf5dde72               0               0
## 6  7a77fe76bb62ff1e46fe2906e1906041               0               0
## 7  7b3fb79d8a7c200c7b15d79ba9fb1f6b               0           33270
## 8  582d1b304d60b4c074d550f3d08af368               0               0
## 9  865d613b55499cf3db6e4477a885676c               0               0
## 10 70351fcb4a655bd2db93506501ca7533               0               0
##    IonT_QC0321WK23 IonT_QC0421WK23
## 1               48              11
## 2                0               0
## 3                0               0
## 4                8               0
## 5                0               0
## 6                6               0
## 7               14            8989
## 8                0               0
## 9                0            4574
## 10               0               0
\end{verbatim}

\begin{Shaded}
\begin{Highlighting}[]
\NormalTok{taxo\_IonTorrent}\OtherTok{\textless{}{-}} \FunctionTok{read.csv}\NormalTok{(}\StringTok{"Raw\_data/IonTorrent/iontorrent\_taxonomy.tsv"}\NormalTok{, }
                        \AttributeTok{header=}\ConstantTok{TRUE}\NormalTok{, }\AttributeTok{sep=}\StringTok{"}\SpecialCharTok{\textbackslash{}t}\StringTok{"}\NormalTok{)}
\FunctionTok{head}\NormalTok{(taxo\_IonTorrent[,}\DecValTok{1}\SpecialCharTok{:}\DecValTok{8}\NormalTok{],}\DecValTok{10}\NormalTok{)}
\end{Highlighting}
\end{Shaded}

\begin{verbatim}
##                                 otu      Kingdom      Phylum            Class
## 1  2949b355cd00039476d9a808f2e76b8e k__Eukaryota p__Oomycota c__Stramenopiles
## 2  9d34744d5f40b8a99cc028bb4908e00b k__Eukaryota p__Oomycota c__Stramenopiles
## 3  61c226ba90e400b577fe1d486db3256d k__Eukaryota p__Oomycota c__Stramenopiles
## 4  af77980eb4e8e0abee3729db262bcdae k__Eukaryota p__Oomycota c__Stramenopiles
## 5  5f838c2663e51903bc2edc67cf5dde72 k__Eukaryota p__Oomycota c__Stramenopiles
## 6  7a77fe76bb62ff1e46fe2906e1906041 k__Eukaryota p__Oomycota c__Stramenopiles
## 7  7b3fb79d8a7c200c7b15d79ba9fb1f6b k__Eukaryota p__Oomycota c__Stramenopiles
## 8  582d1b304d60b4c074d550f3d08af368 k__Eukaryota p__Oomycota c__Stramenopiles
## 9  865d613b55499cf3db6e4477a885676c k__Eukaryota p__Oomycota c__Stramenopiles
## 10 70351fcb4a655bd2db93506501ca7533 k__Eukaryota p__Oomycota c__Stramenopiles
##                Order             Family               Genus
## 1  o__Peronosporales f__Peronosporaceae      g__Peronospora
## 2  o__Peronosporales f__Peronosporaceae      g__Peronospora
## 3  o__Peronosporales f__Peronosporaceae           g__Bremia
## 4  o__Peronosporales f__Peronosporaceae g__Hyaloperonospora
## 5  o__Peronosporales f__Peronosporaceae      g__Peronospora
## 6  o__Peronosporales f__Peronosporaceae       g__Plasmopara
## 7  o__Peronosporales f__Peronosporaceae      g__Peronospora
## 8  o__Peronosporales f__Peronosporaceae      g__Peronospora
## 9  o__Peronosporales f__Peronosporaceae      g__Peronospora
## 10 o__Peronosporales f__Peronosporaceae g__Hyaloperonospora
##                        Species
## 1    s__Peronospora_manshurica
## 2    s__Peronospora_manshurica
## 3          s__Bremia_elliptica
## 4  s__Hyaloperonospora_nesliae
## 5    s__Peronospora_manshurica
## 6       s__Plasmopara_viticola
## 7    s__Peronospora_variabilis
## 8    s__Peronospora_manshurica
## 9  s__Peronospora_boni-henrici
## 10
\end{verbatim}

\begin{Shaded}
\begin{Highlighting}[]
\NormalTok{biom\_IonTorrent }\OtherTok{\textless{}{-}}\NormalTok{ biom\_IonTorrent }\SpecialCharTok{\%\textgreater{}\%}
\NormalTok{  tibble}\SpecialCharTok{::}\FunctionTok{column\_to\_rownames}\NormalTok{(}\StringTok{"otu"}\NormalTok{) }

\NormalTok{taxo\_IonTorrent }\OtherTok{\textless{}{-}}\NormalTok{ taxo\_IonTorrent }\SpecialCharTok{\%\textgreater{}\%}
\NormalTok{  tibble}\SpecialCharTok{::}\FunctionTok{column\_to\_rownames}\NormalTok{(}\StringTok{"otu"}\NormalTok{) }


\NormalTok{biom\_IonTorrent }\OtherTok{\textless{}{-}} \FunctionTok{as.matrix}\NormalTok{(biom\_IonTorrent)}
\NormalTok{taxo\_IonTorrent }\OtherTok{\textless{}{-}} \FunctionTok{as.matrix}\NormalTok{(taxo\_IonTorrent)}

\FunctionTok{class}\NormalTok{(biom\_IonTorrent)}
\end{Highlighting}
\end{Shaded}

\begin{verbatim}
## [1] "matrix" "array"
\end{verbatim}

\begin{Shaded}
\begin{Highlighting}[]
\FunctionTok{class}\NormalTok{(taxo\_IonTorrent) }
\end{Highlighting}
\end{Shaded}

\begin{verbatim}
## [1] "matrix" "array"
\end{verbatim}

\begin{Shaded}
\begin{Highlighting}[]
\NormalTok{OTU\_IonTorrent }\OtherTok{=} \FunctionTok{otu\_table}\NormalTok{(biom\_IonTorrent, }\AttributeTok{taxa\_are\_rows =} \ConstantTok{TRUE}\NormalTok{)}
\NormalTok{TAX\_IonTorrent }\OtherTok{=}\NormalTok{ phyloseq}\SpecialCharTok{::}\FunctionTok{tax\_table}\NormalTok{(taxo\_IonTorrent)}

\NormalTok{IonTorrentPS }\OtherTok{\textless{}{-}} \FunctionTok{phyloseq}\NormalTok{(OTU\_IonTorrent, TAX\_IonTorrent)}
\NormalTok{IonTorrentPS}
\end{Highlighting}
\end{Shaded}

\begin{verbatim}
## phyloseq-class experiment-level object
## otu_table()   OTU Table:         [ 1225 taxa and 94 samples ]
## tax_table()   Taxonomy Table:    [ 1225 taxa by 7 taxonomic ranks ]
\end{verbatim}

And Again for the Nanopore dataset

\begin{Shaded}
\begin{Highlighting}[]
\NormalTok{biom\_Nanopore}\OtherTok{\textless{}{-}} \FunctionTok{read.csv}\NormalTok{(}\StringTok{"Raw\_data/Nanopore/Nanopore\_pseudo\_ASV\_table.csv"}\NormalTok{, }
                        \AttributeTok{header=}\ConstantTok{TRUE}\NormalTok{, }\AttributeTok{sep=}\StringTok{";"}\NormalTok{)}
\FunctionTok{head}\NormalTok{(biom\_Nanopore[,}\DecValTok{1}\SpecialCharTok{:}\DecValTok{5}\NormalTok{],}\DecValTok{10}\NormalTok{)}
\end{Highlighting}
\end{Shaded}

\begin{verbatim}
##           otu plaque_1_its_B1 plaque_1_its_B10 plaque_1_its_B11
## 1  MW633475.1            3394             4832            14977
## 2  AF347031.1            2367              716             3042
## 3  MZ159329.1               8                0                0
## 4  MH855943.1            3759              481             4916
## 5  KU850565.1              19               12               68
## 6  MW709918.1             112                1                1
## 7  MF924721.1               0                0                1
## 8  MG828953.1             458               92              187
## 9  MT310617.1               2                6                0
## 10 JN383490.1               0                0                1
##    plaque_1_its_B12
## 1              1582
## 2               908
## 3                 0
## 4               711
## 5                49
## 6                 2
## 7                 0
## 8                29
## 9                 3
## 10                3
\end{verbatim}

\begin{Shaded}
\begin{Highlighting}[]
\NormalTok{taxo\_Nanopore}\OtherTok{\textless{}{-}} \FunctionTok{read.csv}\NormalTok{(}\StringTok{"Raw\_data/Nanopore/Nanopore\_taxonomy.csv"}\NormalTok{, }
                        \AttributeTok{header=}\ConstantTok{TRUE}\NormalTok{, }\AttributeTok{sep=}\StringTok{","}\NormalTok{)}
\FunctionTok{head}\NormalTok{(taxo\_Nanopore[,}\DecValTok{1}\SpecialCharTok{:}\DecValTok{8}\NormalTok{],}\DecValTok{10}\NormalTok{)}
\end{Highlighting}
\end{Shaded}

\begin{verbatim}
##           otu       Domain           Phylum             Class             Order
## 1  MW633475.1 k__Eukaryota    p__Ascomycota c__dothideomyceta o__Cladosporiales
## 2  AF347031.1 k__Eukaryota    p__Ascomycota c__dothideomyceta   o__Pleosporales
## 3  MZ159329.1 k__Eukaryota p__Basidiomycota   c__Opisthokonta    o__Corticiales
## 4  MH855943.1 k__Eukaryota    p__Ascomycota c__dothideomyceta   o__Pleosporales
## 5  JX089579.1 k__Eukaryota    p__Ascomycota c__dothideomyceta   o__Pleosporales
## 6  JN906978.1 k__Eukaryota    p__Ascomycota c__dothideomyceta o__Cladosporiales
## 7  KU850565.1 k__Eukaryota    p__Ascomycota c__dothideomyceta   o__Pleosporales
## 8  MW709918.1 k__Eukaryota p__Basidiomycota   c__Opisthokonta    o__Tremellales
## 9  MF924721.1 k__Eukaryota p__Basidiomycota   c__Opisthokonta    o__Polyporales
## 10 MG828953.1 k__Eukaryota    p__Ascomycota c__dothideomyceta   o__Pleosporales
##                   Family               Genus                        Species
## 1     f__Cladosporiaceae     g__Cladosporium       s__Cladosporium_herbarum
## 2       f__Pleosporaceae       g__Alternaria        s__Alternaria_alternata
## 3        f__Corticiaceae       g__Phlebiella             s__Phlebiella_vaga
## 4       f__Didymellaceae        g__Epicoccum            s__Epicoccum_nigrum
## 5       f__Pleosporaceae        g__Bipolaris       s__Bipolaris_microstegii
## 6     f__Cladosporiaceae     g__Cladosporium        s__Cladosporium_aphidis
## 7       f__Pleosporaceae      g__Stemphylium      s__Stemphylium_vesicarium
## 8         f__Bulleraceae          g__Bullera                s__Bullera_alba
## 9   f__Incrustoporiaceae     g__Skeletocutis s__Skeletocutis_mopanshanensis
## 10 f__Didymosphaeriaceae g__Pseudopithomyces      s__Pseudopithomyces_rosae
\end{verbatim}

\begin{Shaded}
\begin{Highlighting}[]
\NormalTok{biom\_Nanopore }\OtherTok{\textless{}{-}}\NormalTok{ biom\_Nanopore }\SpecialCharTok{\%\textgreater{}\%}
\NormalTok{  tibble}\SpecialCharTok{::}\FunctionTok{column\_to\_rownames}\NormalTok{(}\StringTok{"otu"}\NormalTok{) }

\NormalTok{taxo\_Nanopore }\OtherTok{\textless{}{-}}\NormalTok{ taxo\_Nanopore }\SpecialCharTok{\%\textgreater{}\%}
\NormalTok{  tibble}\SpecialCharTok{::}\FunctionTok{column\_to\_rownames}\NormalTok{(}\StringTok{"otu"}\NormalTok{) }


\NormalTok{biom\_Nanopore }\OtherTok{\textless{}{-}} \FunctionTok{as.matrix}\NormalTok{(biom\_Nanopore)}
\NormalTok{taxo\_Nanopore }\OtherTok{\textless{}{-}} \FunctionTok{as.matrix}\NormalTok{(taxo\_Nanopore)}

\FunctionTok{class}\NormalTok{(biom\_Nanopore)}
\end{Highlighting}
\end{Shaded}

\begin{verbatim}
## [1] "matrix" "array"
\end{verbatim}

\begin{Shaded}
\begin{Highlighting}[]
\FunctionTok{class}\NormalTok{(taxo\_Nanopore) }
\end{Highlighting}
\end{Shaded}

\begin{verbatim}
## [1] "matrix" "array"
\end{verbatim}

\begin{Shaded}
\begin{Highlighting}[]
\NormalTok{OTU\_Nanopore }\OtherTok{=} \FunctionTok{otu\_table}\NormalTok{(biom\_Nanopore, }\AttributeTok{taxa\_are\_rows =} \ConstantTok{TRUE}\NormalTok{)}
\NormalTok{TAX\_Nanopore }\OtherTok{=}\NormalTok{ phyloseq}\SpecialCharTok{::}\FunctionTok{tax\_table}\NormalTok{(taxo\_Nanopore)}

\NormalTok{NanoporePS }\OtherTok{\textless{}{-}} \FunctionTok{phyloseq}\NormalTok{(OTU\_Nanopore, TAX\_Nanopore)}
\NormalTok{NanoporePS}
\end{Highlighting}
\end{Shaded}

\begin{verbatim}
## phyloseq-class experiment-level object
## otu_table()   OTU Table:         [ 4241 taxa and 94 samples ]
## tax_table()   Taxonomy Table:    [ 4241 taxa by 7 taxonomic ranks ]
\end{verbatim}

\end{document}
